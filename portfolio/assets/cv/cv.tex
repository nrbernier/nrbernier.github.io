%% If you need to pass whatever options to xcolor
\PassOptionsToPackage{dvipsnames}{xcolor}

%% If you are using \orcid or academicons
%% icons, make sure you have the academicons
%% option here, and compile with XeLaTeX
%% or LuaLaTeX.
% \documentclass[10pt,a4paper,academicons]{altacv}

%% Use the "normalphoto" option if you want a normal photo instead of cropped to a circle
% \documentclass[10pt,a4paper,normalphoto]{altacv}

\documentclass[10pt,a4paper,ragged2e]{cvtemplate}

%% AltaCV uses the fontawesome and academicon fonts
%% and packages.
%% See texdoc.net/pkg/fontawecome and http://texdoc.net/pkg/academicons for full list of symbols. You MUST compile with XeLaTeX or LuaLaTeX if you want to use academicons.

% Change the page layout if you need to
\geometry{left=1cm,right=9cm,marginparwidth=6.8cm,marginparsep=1.2cm,top=1.25cm,bottom=1.25cm}

% Change the font if you want to, depending on whether
% you're using pdflatex or xelatex/lualatex
\ifxetexorluatex
  % If using xelatex or lualatex:
  \setmainfont{Lato}
\else
  % If using pdflatex:
  \usepackage[utf8]{inputenc}
  \usepackage[T1]{fontenc}
  \usepackage[default]{lato}
\fi

\usepackage[]{tkz-kiviat} 
\usetikzlibrary{arrows.meta,
                chains,
                positioning}
\usepackage[]{hyperref} 
\usepackage[]{textcomp} 

% Change the colours if you want to
\definecolor{Brownred}{HTML}{a83c09}
%\definecolor{Brownred}{HTML}{8c000f}
%\definecolor{Brownred}{HTML}{a03623}
%\definecolor{Brownred}{HTML}{b04e0f}

\definecolor{Mulberry}{HTML}{72243D}
\definecolor{SlateGrey}{HTML}{2E2E2E}
\definecolor{LightGrey}{HTML}{666666}

%\colorlet{heading}{Sepia}
%\colorlet{heading}{Brownred!180}
%\colorlet{heading}{black}
\colorlet{heading}{SlateGrey}
\colorlet{accent}{Brownred!120}
\colorlet{emphasis}{SlateGrey}
\colorlet{body}{LightGrey}

% Change the bullets for itemize and rating marker
% for \cvskill if you want to
\renewcommand{\itemmarker}{{\small\textbullet}}
\renewcommand{\ratingmarker}{\faCircle}

%% sample.bib contains your publications
%\addbibresource{sample.bib}

\begin{document}
\name{Nathan Bernier}
\tagline{Physicist and aspiring Data Scientist}
\photo{2.8cm}{../images/Photo}
%\personalinfo{%
  % Not all of these are required!
  % You can add your own with \printinfo{symbol}{detail}
  %\email{\href{mailto:nathan.bernier@alumni.epfl.ch}{nathan.bernier@alumni.epfl.ch}}
  %\phone{+41 79 580 87 49}
  %\location{Ch.\ de Passerose 6, 1006 Lausanne, Switzerland}
  %
  %% You MUST add the academicons option to \documentclass, then compile with LuaLaTeX or XeLaTeX, if you want to use \orcid or other academicons commands.
  % \orcid{orcid.org/0000-0000-0000-0000}
%}

%% Make the header extend all the way to the right, if you want.
%\begin{fullwidth}
\makecvheader
\medskip
%\end{fullwidth}

%% Depending on your tastes, you may want to make fonts of itemize environments slightly smaller
% \AtBeginEnvironment{itemize}{\small}

%% Provide the file name containing the sidebar contents as an optional parameter to \cvsection.
%% You can always just use \marginpar{...} if you do
%% not need to align the top of the contents to any
%% \cvsection title in the "main" bar.
\newcommand{\epfl}{\href{http://www.epfl.ch/}{EPFL
(Swiss Federal Institute of Technology Lausanne)}}
\newcommand{\lpqm}{\href{https://k-lab.epfl.ch/}{Laboratory of Quantum Measurements and Photonics}}

\bigskip
\cvsection[cv_sidebar]{Experience}
\timelineperiods{
    2019/{% Postdoc
        \cvevent{Postdoctoral Researcher}%
        {\epfl}{6 months}{}
        %\lpqm
        \begin{itemize}
            \item{Coordination of the research effort within a small team} 
            \item{Mentoring of the junior researchers}
            \item{%Creation of a m
                Module to fit models to data by Bayesian inference}
        \end{itemize}
    },
    2014/
    {% PhD
        \cvevent{Research Assistant}{\epfl}
        %{4 years 7 months}{}
        %{4\textonehalf~years}{}
        {4.5 years}{}
        %\lpqm \\
        %Superconducting 
        Cryogenic
        microwave circuits
        (Quantum technologies)
        \smallskip
        \begin{itemize}
            \item{Experimental results leading to 7 scientific publications
                including one as first author in Nature Physics}
            \item{Development of an original theoretical model 
                confirmed by experiment}
            %\item{Data analysis and visualization for scientific communications}
            %\item{Building of a wiki website for the experiment on the group server}
            \item{
                    %Teaching assistant for several courses, including 
                    Sole teaching assistant for a course on Quantum Optics; 
            substitute for the lecturer on advanced statistics}
            %\item{Main teaching assistant for a course on Quantum Optics,
            %    coaching the students to participate in the IBM Q Experience competition}
        \end{itemize}
    },
    2013/
    {% BU
        \cvevent{Teaching Assistant}
        {\href{http://www.bu.edu/}{Boston University}}
        {2 semesters}{}
        \begin{itemize}
            \item{Several laboratory classes for General Physics}
            \item{Laboratory setup 
            for a new course on Food Science}
        \end{itemize}
    }
   % 2011/
   % {% MJF
   %     \cvevent{Staff Clerk}
   %     {\href{http://www.montreuxjazz.com/2011/en}{Montreux Jazz Festival}}
   %     {4 editions}{}
   %     \begin{itemize} 
   %         \item{Distribution of wages and food vouchers}
   %     \end{itemize}
   % }
}

\bigskip

\cvsection{Education}
\timelineevents{
    2019/{% PhD
        \cvevent{PhD in Physics}{\epfl}{}{}
        Relevant coursework:\\
        Applied machine learning for scientists 
        (sklearn, keras),
        Statistics for physicists (Bayesian inference, with R)
    },
    2013/
    {% Master and bachelor
        \cvevent{MSc \& BSc in Physics}{\epfl}{}{}
        \begin{itemize}
            \item{Master project  at%
                \nameofplace{%
                \href{http://www.harvard.edu/}{Harvard University}},
            leading to article}
            \item{ EPFL \href{https://www.epfl.ch/education/studies/en/financing-study/grants/excellence-fellowships/}
  {Excellence Fellowship} in 2011 and 2012}
            \item{Best average in graduating BSc class (71 students)}
            \item{Exchange year at%
                \nameofplace{%
                \href{http://www.cmu.edu/index.shtml}{Carnegie Mellon University}}
                %($4.00/4.00$ GPA)
            }
            \item{Student assistant for Analytical Mechanics
            and Introduction to Computer Science (C++)}
        \end{itemize}
    }
}
%% If the NEXT page doesn't start with a \cvsection but you'd
%% still like to add a sidebar, then use this command on THIS
%% page to add it. The optional argument lets you pull up the
%% sidebar a bit so that it looks aligned with the top of the
%% main column.
% \addnextpagesidebar[-1ex]{page3sidebar}
\end{document}
